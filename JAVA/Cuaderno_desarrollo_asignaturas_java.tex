\documentclass[]{article}
\usepackage[spanish]{babel}
%opening
\title{PROYECTO JAVA}
\author{ASIGNATURAS}

\begin{document}
	
	\maketitle
	
	\begin{abstract}
		
	\end{abstract}
	
	\section{Objetivos}
	Organizar todos los contenidos del segundo cuatrimestre con ayudas que hagan que el programa sea útil para el estudio y sobretodo para tener una visión amplia del curso.
	Queremos que pueda leer archivos de forma que pueda actualizar nuevos contenidos de forma automática(o si no con un bloc de notas).
	
	\subsection{Ideas de mejora}
	Podemos pasar un int a String y nombrar y referir a vectores con esos nombres para agilizar todo, así no tendremos que referirnos a vectores concretos por su nombre
	\subsection{Segunda subsección}
	Un poco más de texto en el párrafo \textbf{ahora también en negrita (text bold fond)} si nos salimos delas llaves ya no afecta el comando
	
	\section{Funciones principales del programa}
	\section{Progresos}
	\section{Problemas y cambios}
	Estamos haciendo vectores de forma que cada uno son los temas de la asignatura correspondiente, esto hace fácil añadir asignaturas y cambiar los temas de estas. 
	Podríamos haber hecho una matriz con todos los temas, tal vez la información no estaría tan clara pero sería mucho más fácil acceder a ella. En su lugar hemos hecho una clase que al introducirle vectores crea una matriz de la que podemos indicarle la fila y nos la devolverá.
	
	
\end{document}
